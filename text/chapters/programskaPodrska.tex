\chapter{Korištena programska podrška i biblioteke}
\label{ch:podrska}

U ovom poglavlju naveden je popis korištene programske podrške te za svaku stavku kratak opis upotrebe u okviru ovog rada.

\section{Python}
\label{sec:podrskaPython}

Python je interpretirani, objektno orjentirani viši programski jezik. Njegove ugrađene strukture i metode kombinirane s dinamičkim tipiziranjem i povezivanjem čine ga veoma popularnim jezikom za brzi razvoj aplikacija \engl{RAD -- Rapid Application Development} i brzo prototipiranje, kao i za pisanje skripti ili za povezivanje već postojećih komponenti. Python ima jednostavnu sintaksu te je fokusiran prvenstveno na čitljivost koda, čime se olakšava održavanje programa. Podržava module i pakete, čime se potiče na modularno razvijanje programa i ponovno korištenje koda. Python interpreter i standardna biblioteka su dostupni u obliku izvornog koda ili binarne datoteke besplatno i smiju se slobodno distribuirati \citep{Python}.

\subsection{Python Imaging Library (PIL)}
\label{subsec:pil}
\emph{Python Imaging Library} je vanjska biblioteka za Python koja se koristi za učitavanje, konverziju, spremanje i stvaranje novih slika \citep{PIL}

\subsection{pylab}
\label{subsec:pylab}

\emph{PyLab} je modul biblioteke Matplotlib koji služi za iscrtavanje raznih grafičkih elemenata na slikama. U okviru ovog rada koristi se metoda ginput koja vraća točke koje je korisnik odabrao pokazivačem.

\subsection{numpy}
\label{subsec:numpy}

\emph{NumPy} je paket za Python koja dodaje podršku za znastvene proračune. To je biblioteka koja pruža mogućnost rada s višedimenzionalnim nizovima i izvedenim objektima (kao npr. matricama) i sadržio funkcije za brze operacije nad nizovima \citep{NumPy}.

\section{C++}
\label{sec:c++}

\emph{C++} je niži (prema današnjim mjerilima) objektno orjentirani, strogo tipizirani, kompajlirani jezik. Zbog njegove bliskosti hardveru, a i zato što se prevodi u strojni kod u cjelosti prije izvođenja, veoma se brzo izvodi. Sadrži veoma opsežne standardne biblioteke i nebrojeno mnogo vanjskih biblioteka.

\subsection{OpenCV}
\label{subsec:opencv}

\emph{OpenCV (Open source computer vision library)}  je biblioteka otvorenog koda koja pruža mogućnost rada s računalnim vidom i strojnim učenjem. OpenCV je stvoren kako pružio zajedničku infrastukturu za rad s računalnim vidom i kako bi olakšao i ubrzao razvoj softvera koji koristi računalnu percepciju \citep{OpenCV}.