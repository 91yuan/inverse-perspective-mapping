\chapter{Projekcijsko ravninsko preslikavanje}

\section{Prikaz točaka u ravnini homogenom notacijom}

Svaka točka u ravnini u pravokutnom koordinatnom sustavu jednoznačno je određena uređenim parom koordinata $(x, y)$ te je stoga uobičajeno ravninu poistovjetiti s $\mathbb{R}^2$. Zbog toga se točku u ravnini može prikazati vektorom $\textbf{x} = (x, y)^\top$.

Kao što se točka u ravnini može jednoznačno predstaviti  parom koordinata $(x,~y)$, tako se pravac može prikazati jednadžbom $ax + by + c = 0$, gdje se različiti pravci dobivaju mijenjajući parametre $a$, $b$ i $c$. Zbog tog se svojstva pravci mogu bez promjene mogu prikazati u homogenoj notaciji kao stupac-vektor $(a, b, c)^\top$. Ovaj prikaz, doduše, nije jednoznačan, budući da vektori $(a, b, c)$ i $(ka, kb, kc)$ predstavljaju isti pravac za svaki $k \neq 0$, ali za svaki za svaki pravac zapisan u homogenoj notaciji postoji točno jedan pravac zapisan u euklidskoj notaciji \citep{Hartley2004}.

Da bi se točka $\textbf{x} = (x, y)^\top$ zapisala pomoću homogenih koordinata, potrebno je uzeti u obzir slijedeće:
\begin{enumerate}
	\item \label{itm:fst} Ako i samo ako točka $\textbf{x} = (x, y)^\top$ leži na pravcu $\textbf{I} =(a, b, c)^\top$, onda vrijedi jednakost $ax + by + c = 0$.
	\item \ref{itm:fst} se može zapisati kao skalani produkt vektora koji prikazuje točku i homogenog prikaza pravca: $(x, y, 1) \cdot (a, b, c) = (x, y, 1) \cdot \textbf{I} = 0$.
	\item Ako i samo ako vrijedi $(x, y, 1) \cdot \textbf{I} = 0$, onda vrijedi i  $(kx, ky, k) \cdot \textbf{I} = 0$ za bilo koju konstantu $k \neq 0$.
\end{enumerate}

Iz gorenavedenoga vidi se da će točki $\textbf{x} = (x_1, x_2, x_3)^\top$ iz projekcijske ravnine $\mathbb{P}^2$ odgovarati točka $(\frac{x_1}{x_3}, \frac{x_2}{x_3})^\top$ u euklidskoj ravnini $\mathbb{R}^2$, gdje je $x_3$ homogena koordinata točke $\textbf{x}$.

Važno je još napomenuti da se u slučaju kada je homogena koordinata točke $\textbf{x}$ jednaka $0$ kaže da je ta točka u euklidskoj ravnini u beskonačnosti.

\section{Projekcijsko preslikavanje}

\section{Izračun transformacijske matrice}
Izračun transformacijske matrice

\section{Interpolacija}